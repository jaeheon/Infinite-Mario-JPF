\documentclass{article}

\title{JPF/APROP for Infinite Mario}
\author{Chris Lewis and Ben Samuel and Jaeheon Yi}
\begin{document}
\maketitle

\section{Introduction}
Java PathFinder (JPF) is a stateful software model checker developed at NASA AMES Research Center. 
It is a \emph{software model checker} because it systematically explores all paths and inputs directly for software;
it is \emph{stateful} because it saves the program state after every logical operation. 

Although JPF is known as a model checker, it is used more often as a systematic testing and analysis framework. 
Compared with simpler abstractions often used for hardware model checking, software is much more complex with potentially exponentially many more states;
this quality makes exhaustive verification a difficult goal to attain. 
However, the methodical manner in which JPF generates program runs makes it ideal for testing programs; 
this compares much more favorably than with ad-hoc testing, which typically may only explore a fraction of the total state space without ever reaching a failure state. 

JPF is also a powerful analysis framework; for example, one can implement data race and deadlock detection algorithms and have JPF run over different runs to find potentially disastrous program states. 

JPF is a mature system with many extensions; for example, the core JPF system has been combined with symbolic execution and the Bandera model constructor to create a richer feature set for testing and verification. 
JPF has been used to find real errors in Remote Agent Spacecraft Controller, a mission critical software component that had previously deadlocked during operation in space. 
It has also been used to find a very subtle concurrency error in a real time operating system. 

The JPF/APROP project enables creating program property specifications using Java annotations, with listener programs to check that these properties hold during program execution. 
Some annotations that are currently supported include:
\begin{itemize}
\item \tt{@Nonnull} - check 	return values and field for null value
\item \tt{@Const} - check for object modification within scope
\item \tt{@GuardedBy} - check if object is guarded by specified lock
\end{itemize}

\section{What We Tried}

\section{What We Learned}

\section{Conclusion}

\end{document}
